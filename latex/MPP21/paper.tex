\documentclass[conference]{IEEEtran}
%\IEEEoverridecommandlockouts
% The preceding line is only needed to identify funding in the first footnote. If that is unneeded, please comment it out.
\usepackage{cite}
\usepackage{amsmath,amssymb,amsfonts}
\usepackage{algorithmic}
\usepackage{graphicx}
\usepackage{textcomp}
\usepackage{xcolor}
\usepackage{hyperref}
\usepackage{cleveref}
\usepackage{tikz,pgfplots}

% for subfigures
\usepackage[caption=false,font=footnotesize]{subfig}

% for scientific notation like 1e-8
\usepackage{siunitx}
\sisetup{output-exponent-marker=\ensuremath{\mathrm{e}}}

\def\BibTeX{{\rm B\kern-.05em{\sc i\kern-.025em b}\kern-.08em
    T\kern-.1667em\lower.7ex\hbox{E}\kern-.125emX}}

\newcommand*{\email}[1]{\href{mailto:#1}{\nolinkurl{#1}} } 
\newcommand{\dims}[1]{\langle #1 \rangle}

\newcommand{\datafile}{}

\newcommand{\GB}[1]{\textcolor{red}{\textbf{GB}: #1}}
\newcommand{\JW}[1]{\textcolor{blue}{\textbf{JW}: #1}}
\newcommand{\RZ}[1]{\textcolor{purple}{\textbf{RZ}: #1}}

\begin{document}

\title{Accelerating Neural Network Training using Arbitrary Precision Approximating Matrix Multiplication Algorithms
%\thanks{Identify applicable funding agency here. If none, delete this.}
}

\author{\IEEEauthorblockN{Grey Ballard}
\IEEEauthorblockA{\textit{Department of Computer Science} \\
\textit{Wake Forest University}\\
Winston Salem, NC, USA \\
\email{ballard@wfu.edu}}
\and
\IEEEauthorblockN{Jack Weissenberger}
\IEEEauthorblockA{\textit{Department of Computer Science} \\
\textit{Wake Forest University}\\
Winston Salem, NC, USA \\
\email{jack.weissenberger@gmail.com}}
\and
\IEEEauthorblockN{Luoping\GB{?} Zhang}
\IEEEauthorblockA{\textit{Department of Computer Science} \\
\textit{Wake Forest University}\\
Winston Salem, NC, USA \\
\email{zhanl317@wfu.edu}}
}

\maketitle

\begin{abstract}
This document is a model and instructions for \LaTeX.
This and the IEEEtran.cls file define the components of your paper [title, text, heads, etc.]. *CRITICAL: Do Not Use Symbols, Special Characters, Footnotes, 
or Math in Paper Title or Abstract.
\end{abstract}

\begin{IEEEkeywords}
component, formatting, style, styling, insert
\end{IEEEkeywords}

\section{Introduction}

\GB{motivate importance of matmul in NN}

\GB{emphasize unseen potential of APA algorithms}

\GB{find references for low-precision arithmetic inside NN}

\GB{reference this paper: \url{https://ieeexplore.ieee.org/stamp/stamp.jsp?arnumber=1675476} as example use of APA algorithms in practice}


\section{Arbitrary Precision Approximating Algorithms}

\begin{enumerate}
	\item complexity of matmul discussion
	\item describe nature of approximation
	\item give explicit algorithm (Bini?)
	\item table of 12 algorithms
	\item accuracy plot?
\end{enumerate}

\begin{table}
\centering
\caption{Properties of Arbitrary Precision Approximating Algorithms  \\ Speedup and Error are computed assuming 1 recursive step}
\label{tab:algs}
\begin{tabular}{| c | c c c | c c c |} 
\hline
\textbf{Ref} & \textbf{Dims} & \textbf{Rank} & \textbf{Speedup} & $\mathbf{\sigma}$ & $\mathbf{\varphi}$ & \textbf{Error} \\
\hline
- & $\dims{2,2,2}$ & 8 & - & 1 & 0 & \num{1.2e-7} \\
\hline
\cite{BCRL79} & $\dims{3,2,2}$ & 10 & $20\%$ & 1 & 1 & \num{3.5e-4} \\
\cite{Smirnov13} & $\dims{4,2,2}$ & 13 & $23\%$ & 1 & 2 & \num{4.9e-3} \\
\cite{Smirnov13} & $\dims{3,3,2}$ & 14 & $29\%$ & 1 & 3 & \num{1.9e-2} \\
\cite{Smirnov13} & $\dims{5,2,2}$ & 16 & $25\%$ & 1 & 3 & \num{1.9e-2} \\
\cite{Smirnov13} & $\dims{3,3,3}$ & 20 & $35\%$ & 1 & 6 & \num{1.0e-1} \\
\cite{Schonhage81} & $\dims{3,3,3}$ & 21 & $29\%$ & 1 & 2 & \num{4.9e-3} \\
\cite{Smirnov15} & $\dims{7,2,2}$ & 22 & $27\%$ & 1 & 5 & \num{7.0e-2} \\
\cite{Smirnov13} & $\dims{4,4,2}$ & 24 & $33\%$ & 1 & 3 & \num{1.9e-2} \\
\cite{Smirnov16} & $\dims{4,3,3}$ & 27 & $33\%$ & 1 & 3 & \num{1.9e-2} \\
\cite{Smirnov18} & $\dims{5,5,2}$ & 37 & $35\%$ & 1 & 3 & \num{1.9e-2} \\
\cite{Smirnov14} & $\dims{4,4,4}$ & 46 & $39\%$ & 1 & 3 & \num{1.9e-2} \\
\cite{Smirnov18} & $\dims{5,5,5}$ & 90 & $39\%$ & 1 & 3 & \num{1.9e-2} \\
\hline
\end{tabular}
\end{table}

\GB{specify formula for speedup and error values...}

\section{Parallel Fast Matrix Multiplication}

\GB{reference PPoPP paper: \cite{BB15}}

\begin{enumerate}
	\item describe code generation and hybrid parallelization
	\item give performance plots (sequential and parallel)
\end{enumerate}

\section{Neural Network Accuracy}

\begin{enumerate}
	\item describe network (MNIST?) and implementation details
	\item accuracy results in table or fig
\end{enumerate}

\section{Neural Network Performance}

\begin{enumerate}
	\item describe synthetic network and implementation details
	\item perf results for growing dim (maybe for serial and 12-threads separately)
	\item parallel speedup results
\end{enumerate}

\begin{figure}
\subfloat[One thread]{
\centering
\renewcommand{\datafile}{data/NNtime_1t.dat}
%!TEX root = ../paper.tex

\begin{tikzpicture}
\begin{axis}[
	width=.45\textwidth,
	xmode=log,
	xlabel=Dimension, 
	ylabel=Relative Time,
]
\addplot[color=black] table[x=n, y expr=\thisrow{DGEMM}/\thisrow{DGEMM}]{\datafile}; \addlegendentry{DGEMM}
\addplot table[x=n, y expr=\thisrow{Bini}/\thisrow{DGEMM}]{\datafile}; \addlegendentry{Bini}
\end{axis}
\end{tikzpicture}

\label{fig:NNreltime_seq}}
\\
\subfloat[Twelve threads]{
\centering
\renewcommand{\datafile}{data/NNtime_12t.dat}
%!TEX root = ../paper.tex

\begin{tikzpicture}
\begin{axis}[
	width=.45\textwidth,
	xmode=log,
	xlabel=Dimension, 
	ylabel=Relative Time,
]
\addplot[color=black] table[x=n, y expr=\thisrow{DGEMM}/\thisrow{DGEMM}]{\datafile}; \addlegendentry{DGEMM}
\addplot table[x=n, y expr=\thisrow{Bini}/\thisrow{DGEMM}]{\datafile}; \addlegendentry{Bini}
\end{axis}
\end{tikzpicture}

\label{fig:NNreltime_par}}
\caption{Network training time relative to using classical matrix multiplication}
\label{fig:NNreltime}
\end{figure}



\section{Conclusion}

\begin{enumerate}
	\item promising results for synthetic network, goal is to target costly networks bottlenecked by matmul
	\item can tune for matmul dimensions (not just square)
	\item GPU implementation (cite Jianyu Huang's work)
	\item mixed precision fast matmul (APA or otherwise)
\end{enumerate}

\bibliographystyle{IEEEtran}
\bibliography{refs}

\end{document}
