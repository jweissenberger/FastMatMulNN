\documentclass[conference]{IEEEtran}
%\IEEEoverridecommandlockouts
% The preceding line is only needed to identify funding in the first footnote. If that is unneeded, please comment it out.
\usepackage{cite}
\usepackage{amsmath,amssymb,amsfonts}
\usepackage{algorithmic}
\usepackage{graphicx}
\usepackage{textcomp}
\usepackage{xcolor}
\usepackage{ifthen}

% cleveref
\usepackage{hyperref}
\usepackage[nameinlink]{cleveref}
\crefname{section}{\S}{Sections}
\crefname{figure}{Fig.}{Figs.}
\Crefname{figure}{Figure}{Figures}
\crefname{table}{Table}{Tables}

% tikz
\usepackage{tikz,pgfplots}
\usetikzlibrary{positioning}

% for subfigures
\usepackage[caption=false,font=footnotesize]{subfig}

% for scientific notation like 1e-8
\usepackage{siunitx}
\sisetup{output-exponent-marker=\ensuremath{\mathrm{e}}}

\def\BibTeX{{\rm B\kern-.05em{\sc i\kern-.025em b}\kern-.08em
    T\kern-.1667em\lower.7ex\hbox{E}\kern-.125emX}}

\newcommand*{\email}[1]{\href{mailto:#1}{\nolinkurl{#1}} } 
\newcommand{\dims}[1]{\langle #1 \rangle}

% macros for plots
\newcommand{\datafile}{}
\newcommand{\threads}{}
\newcommand{\set}{}

\newcommand{\GB}[1]{\textcolor{red}{\textbf{GB}: #1}}
\newcommand{\JW}[1]{\textcolor{blue}{\textbf{JW}: #1}}
\newcommand{\RZ}[1]{\textcolor{purple}{\textbf{RZ}: #1}}

\begin{document}

\title{Accelerating Neural Network Training using Arbitrary Precision Approximating Matrix Multiplication Algorithms
%\thanks{Identify applicable funding agency here. If none, delete this.}
}

\author{\IEEEauthorblockN{Grey Ballard}
\IEEEauthorblockA{\textit{Department of Computer Science} \\
\textit{Wake Forest University}\\
Winston-Salem, NC, USA \\
\email{ballard@wfu.edu}}
\and
\IEEEauthorblockN{Jack Weissenberger}
\IEEEauthorblockA{\textit{Department of Computer Science} \\
\textit{Wake Forest University}\\
Winston-Salem, NC, USA \\
\email{jack.weissenberger@gmail.com}}
\and
\IEEEauthorblockN{Luoping Zhang}
\IEEEauthorblockA{\textit{Department of Computer Science} \\
\textit{Wake Forest University}\\
Winston-Salem, NC, USA \\
\email{zhanl317@wfu.edu}}
}

\maketitle

\begin{abstract}
Matrix multiplication is one of the bottleneck computations for training the weights within deep neural networks.
To speed up the training phase, we propose to use faster algorithms for matrix multiplication known as Arbitrary Precision Approximating (APA) algorithms.
APA algorithms perform asymptotically fewer arithmetic operations than the classical algorithm, but they compute an approximate result with an error that can be made arbitrarily small in exact arithmetic.
Practical APA algorithms provide significant reduction in computation time and still provide enough accuracy for many applications like neural network training.
We demonstrate that APA algorithms can be efficiently implemented and parallelized for shared-memory platforms to obtain up to 28\% and 21\% speedups over the fastest implementation of the classical algorithm using one core and 12 cores, respectively.
Furthermore, using these algorithms to train a Multi-Layer Perceptron (MLP) network yields no signification reduction in the training or testing error.
Our performance results on a large MLP network show overall performance improvements of up to \GB{W\%}.
\end{abstract}

\section{Introduction}

\emph{Fast} matrix multiplication algorithms are those that perform fewer than the $2n^3+O(n^2)$ flops performed by the classical algorithm.
For example, Strassen's original fast matrix multiplication algorithm performs $O(n^{\log_2 7})$ flops, where $\log_27 \approx 2.81$ \cite{Strassen69}.
The true complexity of matrix multiplication, typically measured as the exponent $\omega$ for complexity $O(n^\omega)$ is an open question, but the current tightest upper bound is 2.37286 \cite{AW21}.
Upper bounds in this range correspond to theoretical algorithms that are not expected to be practical.

One of the reasons such algorithms are impractical is that they are based on so-called \emph{Arbitrary Precision Approximating} (APA) algorithms, which we describe in detail in \cref{sec:APA}.
In exact arithmetic, these algorithms compute an approximation of the correct result, where the error is polynomial in a nonzero parameter of the algorithm \cite{BLR80}.
That is, in exact arithmetic, the error can be made arbitrarily small.
In floating point arithmetic, however, there is a lower bound on the approximation error that depends on the working precision and properties of the algorithm.
Thus, APA algorithms are often considered to have insufficient accuracy for most applications and have largely been overlooked as practical tools despite their performance potential and ability to outperform their exact counterparts \cite{BB15}.

Our goal in this paper is to demonstrate that APA algorithms can offer practical performance improvements for applications that are tolerant to error in matrix multiplications, notably the training phase of neural networks.
Training large neural networks is computationally expensive and the cost has spurred a surge of research into more efficient hardware, better algorithms, and techniques for trading off accuracy for performance.
For instance, low-precision arithmetic has been shown to decrease running time with little to no effect on the ultimate learning task \cite{GAGN15,HCSEB17}, and new floating point formats have been developed and supported in hardware to implement highly efficient low- and mixed-precision computation \cite{KM+19,YWC20}.

Matrix multiplication in particular is a bottleneck computation for many neural networks.
Forward and backward propagation in training the weights of fully connected layers requires matrix multiplication with dimensions given by the sizes of the layers and number of batch samples.
Training convolutional and other types of layers can also be cast as matrix multiplication, either via monolithic multiplications or batches of smaller multiplications \cite{CW+14,GB+19}.
In this paper we focus on Multi-Layer Perceptron (MLP) networks that rely on a sequence of fully connected layers \cite{HSW89}.
Because the sizes of the layers in MLP networks continue to grow, various techniques have been used to reduce the computational demands of the training phase.
For example, low-rank tensor approximation of the weights can reduce both memory and computation \cite{NPOV15}, and fast matrix multiplication (the Strassen-Winograd algorithm) has been applied to the bottleneck matrix multiplications \cite{KAA20}.

Our contribution is the use of APA matrix multiplication algorithms to address this problem and accelerate training.
In particular we
\begin{enumerate}
	\item curate a collection of well-established and recently discovered practical APA algorithms;
	\item extend the framework of \cite{BB15} to generate efficient multithreaded code for all of them, achieving up to 21\% performance improvement over the best parallel classical implementation,
	\item demonstrate the robustness of learning accuracy to approximate matrix multiplications; and
	\item present multithreaded performance improvements of a synthetic MLP of up to \GB{$Y\times$} over the use of classical matrix multiplication.
\end{enumerate}

\section{Practical Arbitrary Precision Approximating Matrix Multiplication Algorithms}
\label{sec:APA}

\subsection{Fast Matrix Multiplication}

Nearly all fast matrix multiplication algorithms are based a rule for multiplying matrices of fixed size, and the reduction in asymptotic complexity stems from using the rule recursively on general matrices.
For example, Strassen's algorithm is specified by a rule for multiplying two $2\times2$ matrices (denoted $\dims{2,2,2}$) using 7 multiplications instead of the classical algorithm's 8 multiplications.
Applying the rule to $n\times n$ matrices, we split each matrix into quadrants, and the 7 multiplications are multiplications of $(n/2)\times (n/2)$ matrices.
For even better efficiency, we can consider larger fixed sizes and find rules that require a lower percentage of multiplications compared to the classical rule.
The number of multiplications in a rule is known as the \emph{rank}, so Strassen's is a rank-7 algorithm.
Algorithms have been derived both analytically and computationally, and there exists a vast set of improvements leading to the current world record \cite{Pan84,CW87,Williams12,AW21}.

\subsection{APA Algorithms}
\label{sec:APAalgs}

A key characteristic of matrix multiplication, first demonstrated by Bini et al.~\cite{BCRL79}, is that it can be approximated to arbitrary accuracy more efficiently than being computed exactly.
An APA algorithm is one that takes as input $A$, $B$, and a scalar parameter $0<\lambda<1$ and computes 
\begin{equation}
\label{eq:APAapprox}
\hat C = A\cdot B + \lambda E + O(\lambda^2).
\end{equation}
Thus, in exact arithmetic, letting $\lambda \rightarrow 0$ achieves arbitrarily small approximation error.
In floating point arithmetic, choosing too small a value for $\lambda$ leads to accumulation of roundoff error that exceeds the approximation error.
We discuss optimizing $\lambda$ in \cref{sec:APAerr}.

In theory, APA algorithms can be converted to exact algorithms at the cost of an extra logarithmic factor of $n$, which is typically hidden by an arbitrarily small increase in the exponent \cite{Bini80}.
In practice, $n$ is not large enough to ignore the logarithmic factor, so we consider each APA algorithm as is.

For a concrete example, we reproduce the rule for the algorithm developed by Bini et al.~\cite{BCRL79} for the $\dims{3,2,2}$ case (multiplying a $3\times 2$ matrix $A$ by a $2\times 2$ matrix $B$), where we use the following notation for input and output matrices:
$$
\begin{bmatrix} A_{11} & A_{12} \\ A_{21} & A_{22} \\ A_{31} & A_{32}  \end{bmatrix} \cdot
\begin{bmatrix} B_{11} & B_{12} \\ B_{21} & B_{22} \end{bmatrix}=
\begin{bmatrix} C_{11} & C_{12} \\ C_{21} & C_{22} \\ C_{31} & C_{32} \end{bmatrix} .$$
Bini's rule is given by
\begin{align*}
M_1 =& (A_{11} + A_{22}) \cdot (\lambda B_{11} + B_{22}) \\
M_2 =& A_{22}\cdot  (-B_{21} - B_{22}) \\
M_3 =& A_{11}\cdot B_{22} \\
M_4 =& (\lambda A_{12} + A_{22})\cdot (-\lambda B_{11} + B_{21}) \\
M_5 =& (A_{11} + \lambda A_{12}) \cdot (\lambda B_{12} + B_{22}) \\
M_6 =& (A_{21} + A_{32}) \cdot (B_{11} + \lambda B_{22}) \\
M_7 =& A_{21} \cdot (-B_{11} - B_{12}) \\
M_8 =& A_{32} \cdot B_{11}\\
M_9 =& (A_{21} + \lambda A_{31}) \cdot (B_{12} - \lambda B_{22})\\
M_{10} =& (\lambda A_{31} + A_{32}) \cdot (B_{12} - \lambda B_{22})\\
\end{align*}
\begin{align*}
C_{11} =& \lambda^{-1} (M_1 + M_2 - M_3 + M_4) \\
C_{12} =& \lambda^{-1} (-M_3 + M_5) \\
C_{21} =& M_4 + M_6 - M_{10}\\
C_{22} =& M_1 - M_5 + M_9\\
C_{31} =& \lambda^{-1} (-M_8 + M_{10}) \\
C_{32} =& \lambda^{-1} (M_6 + M_7 - M_8 + M_9).\\
\end{align*}

We highlight several properties of the rule which are common across all APA algorithms we consider.
First, the rule requires fewer multiplications than the classical one (rank 10 instead of 12 for $\dims{3,2,2}$ here).
Next, each multiplication is between a linear combination of entries of $A$ and a linear combination of entries of $B$, each output entry is computed as a linear combination of the outputs of the multiplications, and each coefficient in the linear combinations is a (Laurent) polynomial in $\lambda$.
The coefficients include both positive and negative powers of $\lambda$, which explains why small values of $\lambda$ can lead to significant roundoff error.
For Bini's algorithm all coefficients are monomial with degree between $-1$ and $1$.

Because of this general pattern, we can encode APA and other fast algorithms succinctly by their linear combination coefficients.
For example, encoding the first multiplication $M_1$ in Bini's algorithm can be done using a triplet of matrices:
\begin{equation}
\label{eq:triplet}
\begin{bmatrix} 1 & 0 \\ 0 & 1 \\ 0 & 0 \end{bmatrix}, \hfill
\begin{bmatrix} \lambda & 0 \\ 0 & 1 \end{bmatrix}, \hfill
\begin{bmatrix} \lambda^{-1} & 0 \\ 0 & 1 \\ 0 & 0 \end{bmatrix}.
\end{equation}
The first two matrices specify the linear combinations taken of entries of $A$ and $B$, and the third matrix specifies the contributions of $M_1$ to the entries of $C$.
Ten such triplets completely specify Bini's algorithm.

\subsection{Numerical Error of APA Algorithms}
\label{sec:APAerr}

In floating point arithmetic, the lower bound on the numerical error of APA algorithms depends on the working precision and two parameters of the algorithm \cite{BLR80}.
The working precision, also referred to as machine precision, is the upper bound on relative error incurred by basic operations in a given floating point format and depends on the number of fractional bits used in the format.
We use the notation $2^{-d}$ for working precision, where $d=52$ for double and $d=23$ for single precision (note that $2^{-52}\approx 10^{-16}$ and $2^{-23}\approx 10^{-7}$).

The two parameters of the APA algorithm specify the contribution of the approximation and roundoff errors, respectively.
The first parameter, $\sigma$, is the smallest positive exponent of the error polynomial and represents the approximation error.
\Cref{eq:APAapprox} shows the error as a polynomial of $\lambda$ whose leading term is linear in $\lambda$.
If an algorithm satisfies \cref{eq:APAapprox} with $E\neq 0$, then $\sigma=1$.
However, if $E=0$, then $\sigma>1$ is the degree of the leading monomial.
Larger $\sigma$ implies smaller error due to the algorithm, though all of the APA algorithms we consider have $\sigma=1$.

The second parameter, $\varphi$, is the largest (in absolute value) negative exponent of the algorithm, computed as the largest sum of negative exponents across all triplets of matrices.
This parameter represents the effect of roundoff error caused by floating point arithmetic involving the largest intermediate values computed by the algorithm.
For example, the triplet given in \cref{eq:triplet} yields a sum of negative exponents of $0+0+1=1$, and in the case of Bini's algorithm, no other triplet has a larger sum, so $\varphi=1$ for that algorithm.
Smaller $\varphi$ implies smaller error due to roundoff, and the APA algorithms we consider exhibit a range of values.

Given these two contributions to the numerical error, $\lambda$ can be optimized to balance the effects based on parameters $\sigma$ and $\varphi$ (and $d$).
As shown by Bini, Lotti, and Romani \cite{BLR80}, the optimal $\lambda$ should be set to $\Theta(2^{-d/(\sigma+\varphi)})$.
Using this value of $\lambda$, the numerical error incurred by the algorithm will be bounded by $O(2^{-d\sigma/(\sigma+\varphi)})$.
Taking Bini's algorithm as an example, we have $\sigma=\varphi=1$, so the error is $O(2^{-d/2})$, or the square root of working precision.
Note that if multiple recursive steps are used, then $\varphi$ increases proportional to the number of steps, so straightforward optimization of $\lambda$ results in error of $O(2^{-d\sigma/(\sigma+s\varphi)})$ for $s$ recursive steps.
The parameters and minimum error for the algorithms we consider are presented in \cref{sec:APAprops}.

We show empirical error results for uniform random inputs of varying dimension in \cref{fig:accuracy}, as compared to the classical algorithm.
We measure the relative Frobenius norm error, or $\|C-\hat C\|_F/\|C\|$, where $\hat C$ is computed by each algorithm and $C$ is computed using the classical algorithm in double precision.
In order to choose the optimal $\lambda$ value for each algorithm, we tested the 5 powers of 2 closest to the theoretical optimal value and chose the best.

\begin{figure}
\centering
\renewcommand{\datafile}{data/matmul_acc.dat}
%!TEX root = ../paper.tex

\begin{tikzpicture}
\begin{axis}[
        width=.45\textwidth,
        ymode=log,
        xlabel=Dimension, 
        ylabel=Relative Error,
]
\addplot[color=black] table[x=n, y=gemm]{\datafile}; \addlegendentry{GEMM}
\addplot table[x=n, y=322]{\datafile}; \addlegendentry{Bini}
\addplot table[x=n, y=333-21]{\datafile}; \addlegendentry{Schonhage}
\addplot table[x=n, y=333-20]{\datafile}; \addlegendentry{Smirnov333}
\end{axis}
\end{tikzpicture}
\caption{Relative Frobenius norm error for APA algorithms on random inputs.}
\label{fig:accuracy}
\end{figure}

Overall, we see little fluctuation of the error over matrix dimension, and the theoretical error bound is an upper bound on all empirical errors.
Note that the legend is ordered according to the error parameters, and the empirical error generally follows this ordering.
The two most accurate APA algorithms are the least recently discovered: Bini's $\dims{3,2,2}$ and Schonhage's $\dims{3,3,3;21}$.
Algorithms that offer more potential speedup tend to be less accurate.
Two algorithms with smaller error than expected are $\dims{5,5,5}$ and ${7,2,2}$, which is explained by the coefficients of those algorithms including fraction pre-factors.
While $\phi$ is computed using the exponents of the largest intermediate term of $\lambda$, the leading term of $\dims{5,5,5}$ has a constant of $1/4$, lessening its magnitude. 
We study the effects of the matrix multiplication error on neural network training and test accuracy in \cref{sec:NNacc}.

\subsection{Practical Algorithms}

We are particularly interested in algorithms with rules for \emph{small} fixed sizes because they have more promise for practical performance.
This is because larger fixed sizes result in multiplications of small submatrices, and matrix multiplication performance degrades for smaller dimensions.
For instance, consider a rule for dimensions $\dims{4,4,4}$ applied to matrices of reasonable size, less than dimension $10{,}000$.
After one recursive call, the submatrices are of size less than 2500, and after two recursive calls, the dimension is less than 625.
At this size, the reduction in number of flops is offset by a reduction in performance, which may result in longer running time.

Instead of focusing on the exponent of the asymptotic complexity of fast algorithms, for practical algorithms we are more interested in the constant reduction in flops of a single recursive level (a single use of the rule of the algorithm).
This is because in practice, for reasonable matrix dimensions (no more than 10,000, say), only 1 or 2 recursive levels will yield performance improvement \cite{BB15}.
We also prefer algorithms with fewer nonzero coefficients in the linear combinations, because while less costly than multiplications, the matrix additions are less efficient (they are memory bandwidth bound) and prevent achieving the ideal speedup given by the reduction in multiplications.
For dimensions $\dims{m,n,k}$ and rank $r$, the ideal speedup for a single recursive step is given by $mnk/r$, and two recursive steps would enable a possible speedup of $(mnk/r)^2$.
These speedups are typically not fully attained because of degradation in performance for smaller matrix dimensions and the overhead of matrix additions.
In the experimental results of this work, we use only 1 recursive step for all algorithms.

\subsection{APA Algorithm Properties}
\label{sec:APAprops}

\Cref{tab:algs} shows the key performance and accuracy properties of the APA algorithms we consider.
Each row corresponds to an algorithm, and the first row includes the classical algorithm for comparison.
For all algorithms, we assume only 1 recursive step is used, though the speedup and error for more steps can be readily calculated from the parameters.
The first column gives the reference where the algorithm was first specified.
The second block column demonstrates the possible performance improvement, where speedup is calculated as $(mnk/r - 1)\cdot100\%$ for $\dims{m,n,k}$ and rank $r$.
The third block column shows error parameters, and the error is calculated as $2^{-d\cdot\sigma/(\sigma+\varphi)}$ with $d=23$, corresponding to single precision.

\begin{table}
\centering
\caption{Properties of Arbitrary Precision Approximating Algorithms  \\ Speedup and Error are computed assuming 1 recursive step}
\label{tab:algs}
\begin{tabular}{| c | c c c | c c c |} 
\hline
\textbf{Ref} & \textbf{Dims} & \textbf{Rank} & \textbf{Speedup} & $\mathbf{\sigma}$ & $\mathbf{\varphi}$ & \textbf{Error} \\
\hline
- & $\dims{2,2,2}$ & 8 & - & 1 & 0 & \num{1.2e-7} \\
\hline
\cite{BCRL79} & $\dims{3,2,2}$ & 10 & $20\%$ & 1 & 1 & \num{3.5e-4} \\
\cite{AS13} & $\dims{4,2,2}$ & 13 & $23\%$ & 1 & 2 & \num{4.9e-3} \\
\cite{Smirnov13} & $\dims{3,3,2}$ & 14 & $29\%$ & 1 & 3 & \num{1.9e-2} \\
\cite{Smirnov13} & $\dims{5,2,2}$ & 16 & $25\%$ & 1 & 3 & \num{1.9e-2} \\
\cite{Smirnov13} & $\dims{3,3,3}$ & 20 & $35\%$ & 1 & 6 & \num{1.0e-1} \\
\cite{Schonhage81} & $\dims{3,3,3}$ & 21 & $29\%$ & 1 & 2 & \num{4.9e-3} \\
\cite{Smirnov15} & $\dims{7,2,2}$ & 22 & $27\%$ & 1 & 5 & \num{7.0e-2} \\
\cite{Smirnov16b} & $\dims{4,4,2}$ & 24 & $33\%$ & 1 & 3 & \num{1.9e-2} \\
\cite{Smirnov16a} & $\dims{4,3,3}$ & 27 & $33\%$ & 1 & 3 & \num{1.9e-2} \\
\cite{Smirnov16b} & $\dims{5,5,2}$ & 37 & $35\%$ & 1 & 3 & \num{1.9e-2} \\
\cite{Smirnov14} & $\dims{4,4,4}$ & 46 & $39\%$ & 1 & 3 & \num{1.9e-2} \\
\cite{Smirnov18} & $\dims{5,5,5}$ & 90 & $39\%$ & 1 & 3 & \num{1.9e-2} \\
\hline
\end{tabular}
\end{table}



\section{Parallel Fast Matrix Multiplication}

We exploit the common algorithmic structure across APA (and exact) algorithms to develop a unified strategy of high performance multithreaded implementation using C++/OpenMP.
We build upon the work of Benson and Ballard \cite{BB15}, using code generation to apply the strategy to each algorithm.

\subsection{Experimental Platform}

All experiments are performed on a dual-socket Intel Xeon E5-2620 (Sandy Bridge) server with 2 sockets each with 6 cores. 
Each socket has a 15 MB L3 cache, and each core has a 256 KB L2 cache and 32 KB L1 data cache. 
Each core has a clock rate of 2.00 GHz (with turbo boost disabled) and peak single precision flop rate of 32 GFLOPS. 
Our code is compiled with GCC version 7.5.0, 
and we use Intel's Math Kernel Library (MKL) version 2019.4.243.

\subsection{Hybrid Parallelization Strategy}

As described in \cref{sec:APAalgs}, each algorithm can be encoded by a set of triplets of coefficient matrices.
From this representation, we generate recursive code that computes the linear combinations for the inputs to each multiplication and combines the outputs.
The linear combinations are computed using a ``write-once'' strategy that was found to be most efficient in terms of memory bandwidth and performance.
The multiplications are either computed recursively or via a call to \texttt{gemm}, the interface to a highly efficient BLAS implementation of the classical algorithm.
Because we use only 1 recursive step in our experiments, there are no recursive calls, and every multiplication is performed by \texttt{gemm}.
We note that the original code generation tool was designed primarily for exact algorithms, and we generalized it to apply to all of the APA algorithms we consider here, some of which have more complicated coefficients than previously considered.

We adopt the ``hybrid'' parallelization strategy proposed by Benson and Ballard \cite{BB15}.
All linear combinations are parallelized in a straightforward way, in order to maximize the memory bandwidth of the machine.
The multiplications are parallelized as follows: given $r$ multiplications and $p$ threads, with $r=p\cdot q + \ell$ for integers $q$ and $\ell<r$, we assign each thread $q$ multiplications to be performed independently, and the remaining $\ell$ multiplications are performed by all threads using the multithreaded implementation of \texttt{gemm}.
\Cref{fig:hybrid} shows the hybrid strategy for $r=10$ (Bini's algorithm) and $p=4$.

\begin{figure}
\centering
%!TEX root = ../paper.tex

\tikzstyle{vertex} = [rectangle,draw,scale=1,node distance=60pt]
\tikzstyle{edge} = [->,line width=1pt]
\tikzstyle{label} = [midway,below]
\def\offset{.1}

\begin{tikzpicture}[scale=.85]

% vertices
\node[vertex,fill=gray!20] (orig) at (0,2) {$C$};
\node[vertex,fill=blue!40] (m1) at (-4.5,0) {$M_1$};
\node[vertex,fill=red!40] (m2) at (-3.5,0) {$M_2$};
\node[vertex,fill=green!40] (m3) at (-2.5,0) {$M_3$};
\node[vertex,fill=yellow!40] (m4) at (-1.5,0) {$M_4$};
\node[vertex,fill=blue!40] (m5) at (-.5,0) {$M_5$};
\node[vertex,fill=red!40] (m6) at (.5,0) {$M_6$};
\node[vertex,fill=green!40] (m7) at (1.5,0) {$M_7$};
\node[vertex,fill=yellow!40] (m8) at (2.5,0) {$M_8$};
\node[vertex,fill=gray!20] (m9) at (3.5,0) {$M_9$};
\node[vertex,fill=gray!20] (m10) at (4.5,0) {$M_{10}$};

% thread number labels
\node[below=\offset of m1] {0};
\node[below=\offset of m2] {1};
\node[below=\offset of m3] {2};
\node[below=\offset of m4] {3};
\node[below=\offset of m5] {0};
\node[below=\offset of m6] {1};
\node[below=\offset of m7] {2};
\node[below=\offset of m8] {3};
\node[below=\offset of m9] {all};
\node[below=\offset of m10] {all};

% edges
\draw[edge] (orig) -- (m1);
\draw[edge] (orig) -- (m2);
\draw[edge] (orig) -- (m3);
\draw[edge] (orig) -- (m4);
\draw[edge] (orig) -- (m5);
\draw[edge] (orig) -- (m6);
\draw[edge] (orig) -- (m7);
\draw[edge] (orig) -- (m8);
\draw[edge] (orig) -- (m9);
\draw[edge] (orig) -- (m10);


\end{tikzpicture}
\caption{Illustration of hybrid parallelization strategy for $r=10$ and 4 threads.  Each thread is assigned two multiplications to compute using single-threaded \texttt{gemm} and the two remaining multiplications are performed using multithreaded \texttt{gemm}.}
\label{fig:hybrid}
\end{figure}

The hybrid strategy is efficient because each thread can achieve close to the peak performance of a core when computing independent matrix multiplications, even for relatively small problems.
The alternative strategy of employing multithreaded \texttt{gemm} for each of the $r$ multiplications (known as ``DFS'') suffers performance degradation for small problems, where the parallel implementation attains a smaller fraction of peak.
The hybrid strategy also perfectly load balances the computation across threads, as opposed to the alternate strategy of assigning the $\ell$ remainder multiplications to $\ell$ different threads (known as ``BFS''), leaving the other $p-\ell$ threads idle.

\subsection{Sequential Performance}

\Cref{fig:matmul_seq} reports the sequential performance of the APA algorithms in comparison to the most efficient implementation of the classical algorithm, MKL's \texttt{sgemm}.
The y-axis is the \emph{effective} GFLOPS, which is measured as $\num{1e-9} \cdot 2n^3 / \text{ time}$.
That is, the GFLOPS reported for APA algorithms is not true performance, as they perform fewer flops than the classical algorithm.
We use this metric to be able to compare relative times across algorithms performing different amounts of computation.
The machine peak for a classical algorithm is given by the horizontal dotted line.

We vary the dimension from 512 up to 8192 and see that all algorithms outperform classical for dimensions larger than 2000 or so.
The highest performing algorithm is $\dims{4,4,4}$, and at dimension 8192, it is 28\% faster than \texttt{sgemm}.
Ignoring the cost of the matrix additions, $\dims{4,4,4}$ performs 39\% fewer flops than the classical algorithm; the drop to 28\% achieved improvement is because of the overhead of matrix additions and the reduced performance of \texttt{sgemm} on smaller matrices.
We note that $\dims{4,4,2}$ is also high performing, achieving a 25\% observed improvement (out of a theoretical 33\%), along with $\dims{3,3,3;20}$ and $\dims{5,5,5}$.

\begin{figure}
\centering
\renewcommand{\datafile}{data/matmul_seq.dat}
\renewcommand{\threads}{1}
%!TEX root = ../paper.tex


\begin{tikzpicture}[scale=.55]
\begin{axis}[
	width=.5\textwidth,
	xlabel=Dimension, 
	ylabel=\ifthenelse{\equal{\threads}{1}}{Effective GFLOPS}{},
	reverse legend,
        legend style={at={(1.02,.5)},anchor=west},
        xtick={512,2048,4096,6144,8192},
        /pgf/number format/.cd, 1000 sep={}
]
\draw[dotted] (axis cs: 512,32*\threads)--(axis cs: 8192,32*\threads);
\addplot[color=black] table[x=n, y=gemm]{\datafile}; \addlegendentry{Classical}
\addplot table[x=n, y=322]{\datafile}; \addlegendentry{$\dims{3,2,2}$}
\addplot table[x=n, y=333-21]{\datafile}; \addlegendentry{$\dims{3,3,3;21}$}
\addplot table[x=n, y=422]{\datafile}; \addlegendentry{$\dims{4,2,2}$}
\addplot table[x=n, y=332]{\datafile}; \addlegendentry{$\dims{3,3,2}$}
\addplot table[x=n, y=522]{\datafile}; \addlegendentry{$\dims{5,2,2}$}
\addplot table[x=n, y=442]{\datafile}; \addlegendentry{$\dims{4,4,2}$}
\addplot table[x=n, y=433]{\datafile}; \addlegendentry{$\dims{4,3,3}$}
\addplot table[x=n, y=552]{\datafile}; \addlegendentry{$\dims{5,5,2}$}
\addplot table[x=n, y=444]{\datafile}; \addlegendentry{$\dims{4,4,4}$}
\addplot table[x=n, y=555]{\datafile}; \addlegendentry{$\dims{5,5,5}$}
\addplot[color=red,mark=triangle] table[x=n, y=722]{\datafile}; \addlegendentry{$\dims{7,2,2}$}
\addplot[color=brown,mark=diamond] table[x=n, y=333-20]{\datafile}; \addlegendentry{$\dims{3,3,3;20}$}
\ifthenelse{\equal{\threads}{12}}{}{\legend{}}
\end{axis}
\end{tikzpicture}

\caption{Single-threaded matrix multiplication performance}
\label{fig:matmul_seq}
\end{figure}

\subsection{Parallel Performance}

We report parallel performance in \cref{fig:matmul_par_6,fig:matmul_par_12} for running with 6 threads (one socket) and 12 threads (both sockets).
Overall, speedups of APA algorithms over classical are reduced in the parallel case.
Again, the theoretical speedup is based on a reduction in the multiplications, and the overhead of additions is the biggest impediment to realizing that speedup.
In the parallel case, the additions can become an even larger bottleneck because the additions are memory bandwidth bound, and the memory bandwidth does not scale with the number of cores \cite{BB15}.
While we can expect close to linear speedup with cores on the multiplications (which are perfectly load balanced), linear scaling is impossible to achieve with the additions.
We also note that when the hybrid method uses all threads on remainder multiplications, the smaller dimensions make it harder for \texttt{sgemm} to scale as well as it can on the original matrix dimensions.

In the performance results for 6 threads (\cref{fig:matmul_par_6}), we see that many of the algorithms start to outperform classical around dimension 2000, though some poor performance is observed for algorithms and particular matrix dimensions.
The fastest algorithms (e.g., $\dims{4,4,4}$ and $\dims{4,4,2}$) achieve a speedup of up to 23\% over \texttt{gemm} and exceed the machine peak for classical algorithms.
We note that the $\dims{4,4,2}$ algorithm has 24 subproblems, which is a multiple of 6, so there are no remainder subproblems that require all threads.

\Cref{fig:matmul_par_12} shows the results for 12 threads.
Here we see a majority of the algorithms are slower than the classical algorithm, even for large matrices.
We attribute the poor performance to the effect of lower \texttt{sgemm} performance for smaller matrices, as well as a lack of NUMA-aware optimization.
The ``ramp-up'' range of \texttt{sgemm} performance is much shallower for 12 threads than for 6 threads, not achieving the plateau performance until dimension 4000 or so.
This implies that the dimension of the submultiplications does not fall on the plateau for any APA algorithm, so the remainder multiplications suffer from poor parallel performance.
The algorithm with no remainder multiplications, $\dims{4,2,2}$, does not suffer this problem and maintains higher performance.
It exceeds \texttt{sgemm}'s effective performance at dimension 4000, exceeds the peak parallel performance of any classical algorithm for larger dimensions, and achieves a 21\% speedup over \texttt{sgemm} at dimension 8192 for an effective rate of 389 GFLOPS.

\begin{figure}
\centering
\renewcommand{\datafile}{data/matmul_par_6.dat}
\renewcommand{\threads}{6}
%!TEX root = ../paper.tex


\begin{tikzpicture}[scale=.55]
\begin{axis}[
	width=.5\textwidth,
	xlabel=Dimension, 
	ylabel=\ifthenelse{\equal{\threads}{1}}{Effective GFLOPS}{},
	reverse legend,
        legend style={at={(1.02,.5)},anchor=west},
        xtick={512,2048,4096,6144,8192},
        /pgf/number format/.cd, 1000 sep={}
]
\draw[dotted] (axis cs: 512,32*\threads)--(axis cs: 8192,32*\threads);
\addplot[color=black] table[x=n, y=gemm]{\datafile}; \addlegendentry{Classical}
\addplot table[x=n, y=322]{\datafile}; \addlegendentry{$\dims{3,2,2}$}
\addplot table[x=n, y=333-21]{\datafile}; \addlegendentry{$\dims{3,3,3;21}$}
\addplot table[x=n, y=422]{\datafile}; \addlegendentry{$\dims{4,2,2}$}
\addplot table[x=n, y=332]{\datafile}; \addlegendentry{$\dims{3,3,2}$}
\addplot table[x=n, y=522]{\datafile}; \addlegendentry{$\dims{5,2,2}$}
\addplot table[x=n, y=442]{\datafile}; \addlegendentry{$\dims{4,4,2}$}
\addplot table[x=n, y=433]{\datafile}; \addlegendentry{$\dims{4,3,3}$}
\addplot table[x=n, y=552]{\datafile}; \addlegendentry{$\dims{5,5,2}$}
\addplot table[x=n, y=444]{\datafile}; \addlegendentry{$\dims{4,4,4}$}
\addplot table[x=n, y=555]{\datafile}; \addlegendentry{$\dims{5,5,5}$}
\addplot[color=red,mark=triangle] table[x=n, y=722]{\datafile}; \addlegendentry{$\dims{7,2,2}$}
\addplot[color=brown,mark=diamond] table[x=n, y=333-20]{\datafile}; \addlegendentry{$\dims{3,3,3;20}$}
\ifthenelse{\equal{\threads}{12}}{}{\legend{}}
\end{axis}
\end{tikzpicture}

\caption{Multithreaded matrix multiplication performance (6 threads)}
\label{fig:matmul_par_6}
\end{figure}

\begin{figure}
\centering
\renewcommand{\datafile}{data/matmul_par_12.dat}
\renewcommand{\threads}{12}
%!TEX root = ../paper.tex


\begin{tikzpicture}[scale=.55]
\begin{axis}[
	width=.5\textwidth,
	xlabel=Dimension, 
	ylabel=\ifthenelse{\equal{\threads}{1}}{Effective GFLOPS}{},
	reverse legend,
        legend style={at={(1.02,.5)},anchor=west},
        xtick={512,2048,4096,6144,8192},
        /pgf/number format/.cd, 1000 sep={}
]
\draw[dotted] (axis cs: 512,32*\threads)--(axis cs: 8192,32*\threads);
\addplot[color=black] table[x=n, y=gemm]{\datafile}; \addlegendentry{Classical}
\addplot table[x=n, y=322]{\datafile}; \addlegendentry{$\dims{3,2,2}$}
\addplot table[x=n, y=333-21]{\datafile}; \addlegendentry{$\dims{3,3,3;21}$}
\addplot table[x=n, y=422]{\datafile}; \addlegendentry{$\dims{4,2,2}$}
\addplot table[x=n, y=332]{\datafile}; \addlegendentry{$\dims{3,3,2}$}
\addplot table[x=n, y=522]{\datafile}; \addlegendentry{$\dims{5,2,2}$}
\addplot table[x=n, y=442]{\datafile}; \addlegendentry{$\dims{4,4,2}$}
\addplot table[x=n, y=433]{\datafile}; \addlegendentry{$\dims{4,3,3}$}
\addplot table[x=n, y=552]{\datafile}; \addlegendentry{$\dims{5,5,2}$}
\addplot table[x=n, y=444]{\datafile}; \addlegendentry{$\dims{4,4,4}$}
\addplot table[x=n, y=555]{\datafile}; \addlegendentry{$\dims{5,5,5}$}
\addplot[color=red,mark=triangle] table[x=n, y=722]{\datafile}; \addlegendentry{$\dims{7,2,2}$}
\addplot[color=brown,mark=diamond] table[x=n, y=333-20]{\datafile}; \addlegendentry{$\dims{3,3,3;20}$}
\ifthenelse{\equal{\threads}{12}}{}{\legend{}}
\end{axis}
\end{tikzpicture}

\caption{Multithreaded matrix multiplication performance (12 threads)}
\label{fig:matmul_par_12}
\end{figure}

\section{Neural Network Accuracy}
\label{sec:NNacc}

DRAFT: To measure the effect of the error introduced from the APA algorithms, we trained a simple feed forward network on the MNIST dataset and measured its performance over the course of training. MNIST is a common machine learning dataset of 70,000 images of hand-written numerical digits. Each of the images are composed of 28 by 28 grayscale pixels. To feed the images into the network we flatten each of the images into a 784 dimensional vector. The dataset was split into 60,000 examples for the training set and 10,000 examples for the test set. The network is composed of 4 fully-connected layers of shapes 784, 300, 300, and 10.  The input layer takes in the values of all of the pixels which is why it 784 nodes and the output layer is 10 nodes, one for each digit 0-9. The network was created using Tensorflow version 2.2.0.
APA algorithms were used for the middle multiplication of the network (300x300). The input and output layers used the classical matrix multiplication algorithm. One network was trained for 50 epochs for each of the APA algorithms and one was trained using the classical matrix multiplication algorithm.
Figure 6 plots the accuracy on the training data over each epoch and figure 7 plots the test accuracy over each epoch. The error introduced from these algorithms did not have much of an effect on these models and they were able to reach the same network performance as the classical algorithm.

\begin{figure}
\centering
\renewcommand{\datafile}{data/NN_acc.dat}
\renewcommand{\set}{train}
%!TEX root = ../paper.tex


\begin{tikzpicture}[scale=.8]
\begin{axis}[
	width=.5\textwidth,
	xlabel=Epoch, 
	ylabel=\ifthenelse{\equal{\set}{test}}{}{Accuracy}, 
        legend style={at={(1.02,.5)},anchor=west},
        xtick={0,10,20,30,40,50},
        ymin={.94},
        ymax={1},
        reverse legend,
]
\addplot[color=black] table[x=epoch, y=dgemm_\set]{\datafile}; \addlegendentry{Classical}
\addplot table[x=epoch, y=bini322_\set]{\datafile}; \addlegendentry{$\dims{3,2,2}$}
\addplot table[x=epoch, y=schonhage333_\set]{\datafile}; \addlegendentry{$\dims{3,3,3;21}$}
\addplot table[x=epoch, y=smirnov224_\set]{\datafile}; \addlegendentry{$\dims{4,2,2}$}
\addplot table[x=epoch, y=smirnov323_\set]{\datafile}; \addlegendentry{$\dims{3,3,2}$}
\addplot table[x=epoch, y=smirnov225_\set]{\datafile}; \addlegendentry{$\dims{5,2,2}$}
\addplot table[x=epoch, y=smirnov442_\set]{\datafile}; \addlegendentry{$\dims{4,4,2}$}
\addplot table[x=epoch, y=smirnov334_\set]{\datafile}; \addlegendentry{$\dims{4,3,3}$}
\addplot table[x=epoch, y=smirnov552_\set]{\datafile}; \addlegendentry{$\dims{5,5,2}$}
\addplot table[x=epoch, y=smirnov444_\set]{\datafile}; \addlegendentry{$\dims{4,4,4}$}
\addplot table[x=epoch, y=smirnov555_\set]{\datafile}; \addlegendentry{$\dims{5,5,5}$}
\addplot[color=red,mark=triangle] table[x=epoch, y=smirnov272_\set]{\datafile}; \addlegendentry{$\dims{7,2,2}$}
\addplot[color=brown,mark=diamond] table[x=epoch, y=smirnov333_\set]{\datafile}; \addlegendentry{$\dims{3,3,3;20}$}
\ifthenelse{\equal{\set}{test}}{}{\legend{}}
\end{axis}
\end{tikzpicture}

\caption{MLP training accuracy for MNIST data set}
\label{fig:matmul_seq}
\end{figure}

\begin{figure}
\centering
\renewcommand{\datafile}{data/NN_acc.dat}
\renewcommand{\set}{test}
%!TEX root = ../paper.tex


\begin{tikzpicture}[scale=.8]
\begin{axis}[
	width=.5\textwidth,
	xlabel=Epoch, 
	ylabel=\ifthenelse{\equal{\set}{test}}{}{Accuracy}, 
        legend style={at={(1.02,.5)},anchor=west},
        xtick={0,10,20,30,40,50},
        ymin={.94},
        ymax={1},
        reverse legend,
]
\addplot[color=black] table[x=epoch, y=dgemm_\set]{\datafile}; \addlegendentry{Classical}
\addplot table[x=epoch, y=bini322_\set]{\datafile}; \addlegendentry{$\dims{3,2,2}$}
\addplot table[x=epoch, y=schonhage333_\set]{\datafile}; \addlegendentry{$\dims{3,3,3;21}$}
\addplot table[x=epoch, y=smirnov224_\set]{\datafile}; \addlegendentry{$\dims{4,2,2}$}
\addplot table[x=epoch, y=smirnov323_\set]{\datafile}; \addlegendentry{$\dims{3,3,2}$}
\addplot table[x=epoch, y=smirnov225_\set]{\datafile}; \addlegendentry{$\dims{5,2,2}$}
\addplot table[x=epoch, y=smirnov442_\set]{\datafile}; \addlegendentry{$\dims{4,4,2}$}
\addplot table[x=epoch, y=smirnov334_\set]{\datafile}; \addlegendentry{$\dims{4,3,3}$}
\addplot table[x=epoch, y=smirnov552_\set]{\datafile}; \addlegendentry{$\dims{5,5,2}$}
\addplot table[x=epoch, y=smirnov444_\set]{\datafile}; \addlegendentry{$\dims{4,4,4}$}
\addplot table[x=epoch, y=smirnov555_\set]{\datafile}; \addlegendentry{$\dims{5,5,5}$}
\addplot[color=red,mark=triangle] table[x=epoch, y=smirnov272_\set]{\datafile}; \addlegendentry{$\dims{7,2,2}$}
\addplot[color=brown,mark=diamond] table[x=epoch, y=smirnov333_\set]{\datafile}; \addlegendentry{$\dims{3,3,3;20}$}
\ifthenelse{\equal{\set}{test}}{}{\legend{}}
\end{axis}
\end{tikzpicture}

\caption{MLP test accuracy for MNIST data set}
\label{fig:matmul_seq}
\end{figure}

\section{Neural Network Performance}
\label{sec:NNperf}

\begin{enumerate}
	\item describe synthetic network and implementation details
	\item perf results for growing dim (maybe for serial and 12-threads separately)
	\item parallel speedup results
\end{enumerate}

\begin{figure}
\subfloat[One thread]{
\centering
\renewcommand{\datafile}{data/NN_seq.dat}
%!TEX root = ../paper.tex

\begin{tikzpicture}[scale=.55]
\begin{axis}[
	width=.5\textwidth,
	xmode=log,
	log basis x={2},
	xlabel=Dimension, 
	ylabel=\ifthenelse{\equal{\threads}{1}}{Relative Time}{},
	legend style={at={(1.02,.5)},anchor=west},
	xtick={512,1024,2048,4096,8192},
        /pgf/number format/.cd, 1000 sep={},
        reverse legend,
]
\addplot[color=black] table[x=n, y expr=\thisrow{dgemm}/\thisrow{dgemm}]{\datafile}; \addlegendentry{Classical}
\addplot table[x=n, y expr=\thisrow{dgemm}/\thisrow{bini322}]{\datafile}; \addlegendentry{$\dims{3,2,2}$}
\addplot table[x=n, y expr=\thisrow{dgemm}/\thisrow{schonhage333}]{\datafile}; \addlegendentry{$\dims{3,3,3;21}$}
\addplot table[x=n, y expr=\thisrow{dgemm}/\thisrow{smirnov224}]{\datafile}; \addlegendentry{$\dims{4,2,2}$}
\addplot table[x=n, y expr=\thisrow{dgemm}/\thisrow{smirnov323}]{\datafile}; \addlegendentry{$\dims{3,3,2}$}
\addplot table[x=n, y expr=\thisrow{dgemm}/\thisrow{smirnov225}]{\datafile}; \addlegendentry{$\dims{5,2,2}$}
\addplot table[x=n, y expr=\thisrow{dgemm}/\thisrow{smirnov442}]{\datafile}; \addlegendentry{$\dims{4,4,2}$}
\addplot table[x=n, y expr=\thisrow{dgemm}/\thisrow{smirnov334}]{\datafile}; \addlegendentry{$\dims{4,3,3}$}
\addplot table[x=n, y expr=\thisrow{dgemm}/\thisrow{smirnov552}]{\datafile}; \addlegendentry{$\dims{5,5,2}$}
\addplot table[x=n, y expr=\thisrow{dgemm}/\thisrow{smirnov444}]{\datafile}; \addlegendentry{$\dims{4,4,4}$}
\addplot table[x=n, y expr=\thisrow{dgemm}/\thisrow{smirnov555}]{\datafile}; \addlegendentry{$\dims{5,5,5}$}
\addplot[color=red,mark=triangle] table[x=n, y expr=\thisrow{dgemm}/\thisrow{smirnov272}]{\datafile}; \addlegendentry{$\dims{7,2,2}$}
\addplot[color=brown,mark=diamond] table[x=n, y expr=\thisrow{dgemm}/\thisrow{smirnov333}]{\datafile}; \addlegendentry{$\dims{3,3,3;20}$}
\ifthenelse{\equal{\threads}{12}}{}{\legend{}}
\end{axis}
\end{tikzpicture}

\label{fig:NNreltime_seq}}
\\
\subfloat[Twelve threads]{
\centering
\renewcommand{\datafile}{data/NNtime_12t.dat}
%%!TEX root = ../paper.tex

\begin{tikzpicture}[scale=.55]
\begin{axis}[
	width=.5\textwidth,
	xmode=log,
	log basis x={2},
	xlabel=Dimension, 
	ylabel=\ifthenelse{\equal{\threads}{1}}{Relative Time}{},
	legend style={at={(1.02,.5)},anchor=west},
	xtick={512,1024,2048,4096,8192},
        /pgf/number format/.cd, 1000 sep={},
        reverse legend,
]
\addplot[color=black] table[x=n, y expr=\thisrow{dgemm}/\thisrow{dgemm}]{\datafile}; \addlegendentry{Classical}
\addplot table[x=n, y expr=\thisrow{dgemm}/\thisrow{bini322}]{\datafile}; \addlegendentry{$\dims{3,2,2}$}
\addplot table[x=n, y expr=\thisrow{dgemm}/\thisrow{schonhage333}]{\datafile}; \addlegendentry{$\dims{3,3,3;21}$}
\addplot table[x=n, y expr=\thisrow{dgemm}/\thisrow{smirnov224}]{\datafile}; \addlegendentry{$\dims{4,2,2}$}
\addplot table[x=n, y expr=\thisrow{dgemm}/\thisrow{smirnov323}]{\datafile}; \addlegendentry{$\dims{3,3,2}$}
\addplot table[x=n, y expr=\thisrow{dgemm}/\thisrow{smirnov225}]{\datafile}; \addlegendentry{$\dims{5,2,2}$}
\addplot table[x=n, y expr=\thisrow{dgemm}/\thisrow{smirnov442}]{\datafile}; \addlegendentry{$\dims{4,4,2}$}
\addplot table[x=n, y expr=\thisrow{dgemm}/\thisrow{smirnov334}]{\datafile}; \addlegendentry{$\dims{4,3,3}$}
\addplot table[x=n, y expr=\thisrow{dgemm}/\thisrow{smirnov552}]{\datafile}; \addlegendentry{$\dims{5,5,2}$}
\addplot table[x=n, y expr=\thisrow{dgemm}/\thisrow{smirnov444}]{\datafile}; \addlegendentry{$\dims{4,4,4}$}
\addplot table[x=n, y expr=\thisrow{dgemm}/\thisrow{smirnov555}]{\datafile}; \addlegendentry{$\dims{5,5,5}$}
\addplot[color=red,mark=triangle] table[x=n, y expr=\thisrow{dgemm}/\thisrow{smirnov272}]{\datafile}; \addlegendentry{$\dims{7,2,2}$}
\addplot[color=brown,mark=diamond] table[x=n, y expr=\thisrow{dgemm}/\thisrow{smirnov333}]{\datafile}; \addlegendentry{$\dims{3,3,3;20}$}
\ifthenelse{\equal{\threads}{12}}{}{\legend{}}
\end{axis}
\end{tikzpicture}

\label{fig:NNreltime_par}}
\caption{Network training time relative to using classical matrix multiplication}
\label{fig:NNreltime}
\end{figure}



\section{Conclusion}

\begin{enumerate}
	\item promising results for synthetic network, goal is to target costly networks bottlenecked by matmul (\GB{\cite{NPOV15} uses a network including layers with dimensions 25K x 4K and 4K x 4K})
	\item can tune for matmul dimensions (not just square)
	\item GPU implementation (cite Jianyu Huang's work)
	\item mixed precision fast matmul (APA or otherwise)
\end{enumerate}

\bibliographystyle{IEEEtran}
\bibliography{refs}

\end{document}
